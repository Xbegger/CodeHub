\documentclass{article}  
\usepackage{ctex}
\usepackage{amsmath}

\usepackage{geometry}

\geometry{a4paper,left=1cm,right=1cm,top=1cm,bottom=1cm}
\setlength{\parindent}{12pt} %自然段第一行的缩进量为12pt
\setlength{\parskip}{10pt plus1pt minus1pt}
%自然段之间的距离为10pt,并可在8pt到11pt之间变化
\setlength{\baselineskip}{8pt plus2pt minus1pt}
%行间距为8pt,并可在7pt到10pt之间变化
\setlength{\textheight}{21true cm}%版面高为21厘米
\setlength{\textwidth}{14.5true cm}%版面宽为14.5厘米



\begin{document}
    1.开始接受服务时间服从参数为$\lambda$的泊松分布过程,接受服务的的时长服从参数为$\lambda$的指数分布。\\
    已知每个人接受服务的时间是独立的并服从均值为20分钟的指数分布
    \begin{gather*}
        \frac{1}{\lambda} = 20 \Rightarrow  \lambda = \frac{1}{20}, 
        P(N(t) = n) = e^{-\frac{t}{20}} \frac{(\frac{t}{20})^n}{n!},E(N(t))=\frac{t}{20} \\
        E(N(4 \times 60 ))= 12, P(N(4 * 60) = 9) = e^{-12} \frac{12^9}{9!} \approx 0.0874  
    \end{gather*}
    2.已知天文台观测到的流星流是一个泊松分布过程,且平均每小时观察到3颗流星\\
    \begin{gather*}
        \Rightarrow E(N(1))= \lambda 1 = 3 ,\lambda = 3,P(N(t) = n) = e^{-3t} \frac{(3t)^n}{n!}\\
        P(N(4) = 0) = e^{-3 \times 4} \frac{(3 \times 4)^0}{0!} = e^{-12}
    \end{gather*}
    3.已知乘客按照强度为$\lambda$强度的泊松分布过程到达火车站\\
    \begin{gather*}
        \Rightarrow P(N(t) = n) = e^{-\lambda t} \frac{(\lambda t)^n}{n!}\\
        E(\sum_{i=1}^{N(t)}(t-T_i)) = \sum_{n=0}^{+\infty} E(\sum_{i=1}^{N(t)}(t-T_i) | N(t)=n)P(N(t) =n)\\
        =\sum_{n=0}^{+\infty} E(\sum_{i=1}^{n} (t-T_i)) e^{-\lambda t} \frac{(\lambda t)^n}{n!}
        =\sum_{n=0}^{+\infty} (n\cdot t - \sum_{i=1}^{n} E(T_i)) e^{-\lambda t} \frac{(\lambda t)^n}{n!}\\
        =\sum_{n=0}^{+\infty} (n \cdot  t - n \cdot \lambda \cdot t)  e^{-\lambda t} \frac{(\lambda t)^n}{n!}
        = t \cdot (1-\lambda) \cdot  e^{-\lambda t} \sum_{n=0}^{+\infty} n \cdot \frac{(\lambda t)^n}{n!}\\
        = t \cdot (1 - \lambda) \cdot e^{-\lambda t} \cdot (\lambda t) \cdot e^{\lambda t}
        = \lambda t^2 (1 - \lambda)
    \end{gather*}
    4.已知顾客以6分钟的平均速率进入某商场,这一过程可以用泊松分布过程来描述。
    \begin{gather*}
        \Rightarrow \text{设$N_1(t)$表示在时间$(0,t])$进入商城的人数,服从强度为6的泊松分布过程}\\
        P(N_1(t) = n) = e^{-6t} \frac{(6t)^n}{n!}
    \end{gather*}
    已知进入该商场的每位顾客买东西的概率为0.9
    \begin{gather*}
        \Rightarrow \text{设$Y(i)$表示第i位顾客是否购买东西,1表示买,0表示不买;}
        \text{$Y(i)$服从于概率为0.9的$0-1$分布} \\  
        \text{设$X(t)$表示t时刻在商场买东西的顾客数}, X(t) = \sum_{i=0}^{N_1(t)} Y(i)\\
        E(X(t)) = \sum^{+\infty}_{n = 0} E(X(t) | N_1(t)=n) P(N_1(t)=n)
        =\sum_{n=0} ^{+\infty}E(\sum_{i=0}^{n}Y(i))e^{-6t}\frac{(6t)^n}{n!}
        = e^{-6t}\sum_{n=0}^{+\infty} n \cdot 0.9 \cdot \frac{(6t)^n}{n!}\\
        = e^{-6t} \cdot 0.9 \cdot (6t) \cdot e^{6t} = 5.4t 
    \end{gather*}
\end{document}