\documentclass{article}  
\usepackage{ctex}
\usepackage{amsmath}

\usepackage{geometry}

\geometry{a4paper,left=1cm,right=1cm,top=1cm,bottom=1cm}
\setlength{\parindent}{12pt} %自然段第一行的缩进量为12pt
\setlength{\parskip}{10pt plus1pt minus1pt}
%自然段之间的距离为10pt,并可在8pt到11pt之间变化
\setlength{\baselineskip}{8pt plus2pt minus1pt}
%行间距为8pt,并可在7pt到10pt之间变化
\setlength{\textheight}{21true cm}%版面高为21厘米
\setlength{\textwidth}{14.5true cm}%版面宽为14.5厘米



\begin{document}
    1.
    (1)已知甲盒子有两个红球,乙盒子有四个白球。每次从两个盒子各取一球交换\\
    $P(X_n|X_{n-1}, X_{n-1}, \cdots, X_1, X_0) = P(X_n|X_{n-1})$
    所以随机过程$\{X_n,n=0,1,\cdots\}$是一个Markov链。\\
    转移概率矩阵为
    $\begin{matrix}
                    & 0个红球 & 1个红球 & 2个红球 \\
            0个红球  & \frac{1}{2} & \frac{1}{2} & 0\\
            1个红球  & \frac{3}{8} & \frac{1}{2}  & \frac{1}{8} \\
            2个红球  & 0  &  1  & 0
    \end{matrix}$\\
    (2) 由(1)知转移概率矩阵$
    P = \begin{matrix}
            & 0个红球 & 1个红球 & 2个红球 \\
            0个红球  & \frac{1}{2} & \frac{1}{2} & 0\\
            1个红球  & \frac{3}{8} & \frac{1}{2}  & \frac{1}{8} \\
            2个红球  & 0  &  1  & 0
        \end{matrix}$。
    从转移概率矩阵可知,各个状态之间是互通是不可约的,所以随机过程是正常返的,又因为周期为1,
    所以随机过程$\{X_n, n=0,1, \cdots\}$是遍历的\\
    (3)
    \begin{gather*}
        \begin{cases}
            \pi_1 = \pi_1 \times \frac{1}{2} + \pi_2 \times \frac{1}{2}\\
            \pi_2 = \pi_1 \times \frac{3}{8} + \pi_2 \times \frac{1}{2} + \pi_3 \times \frac{1}{8}\\
            \pi_3 = \pi_2 \times 1\\
            \sum_{i=1}^{3} \pi_i = 1
        \end{cases}
    \end{gather*}  
    解得$\pi_1 = \pi_2 = \pi_3 = \frac{1}{3}$
    2.
    \begin{gather*}
        \text{令$A = \{ X_1, X_2,\cdots, X_5\}$} 
        P(A) = \sum_i P(X_2, X_3, X_4, X_5| X_1 = i) \cdot P(X_1 = i)\\   
        = \sum_i \sum_j P(X_3, X_4, X_5| X_2=j, X_1 = i) \cdot P(X_1 =i) \cdot P(X_2=j|X_1=i)\\
        = \sum \cdots \sum P(X_5|X_4, X_3, X_2, X_1) \cdot P(X_4|X_3, X_2, X_1) \cdot P(X_3|X_2,X_1) \cdot P(X_2|X_1) \cdot P(X_1)
    \end{gather*}
    甲盒中:红球90个,白球10个,摸到一个球后放回另一个颜色的球。当前摸到红球的概率只受前一时刻摸球的影响,是一个马尔可夫链\\
    $P(A) = \sum \cdots \sum P(X_5|X_4)\cdot P(X4|X_3)\cdot P(X_3|X_2)\cdot P(X_2|X_1)\cdot P(X_1)$\\
    所以
    $$P(X_1=\text{红}, X_2 =\text{红}, X_3 = \text{红}, X_4 = \text{红}, X_5 = \text{白} )
    = 0.9 * 0.89 * 0.88 * 0.87 * 0.14 = 0.8585
    $$\\
    乙盒:红球50,白球50个,每次摸到球后放回。每次摸到红球的概率互相独立\\
    $P(A) = P(X_1) \cdot P(X_2) \cdot P(X_3) \cdot P(X_4) \cdot P(X_5)$
    所以
    $$P(X_1=\text{红}, X_2 =\text{红}, X_3 = \text{红}, X_4 = \text{红}, X_5 = \text{白} )
    = 0.5 * 0.5 * 0.5 *0.5 * 0.5 = 0.3125
    $$\\
    丙盒:红球40个,白球60个,每次摸到球后不放回。每次摸到红球的概率受前一时刻摸球的影响,是一个马尔可夫链\\
    $P(A) = \sum \cdots \sum P(X_5|X_4)\cdot P(X4|X_3)\cdot P(X_3|X_2)\cdot P(X_2|X_1)\cdot P(X_1)$\\
    所以
    $P(X_1=\text{红}, X_2 =\text{红}, X_3 = \text{红}, X_4 = \text{红}, X_5 = \text{白} )
    = \frac{40}{100} \times \frac{39}{99} \times \frac{38}{98} \times \frac{37}{97} \times \frac{60}{96}
    = 0.1456
    $

    \begin{gather*}
        P(\text{甲}|A) = \frac{P(A|\text{甲}) \cdot P(\text{甲})}{P(A)} 
        = \frac{P(A|\text{甲}) \cdot P(\text{甲}) }{\sum_{\omega} P(A|\omega) P(\omega)}\\
        = 0.6521\\
        \text{同理可得}
        P(\text{乙}|A) = 0.2373
        P(\text{丙}|A) = 0.1106
    \end{gather*}
    所以来自甲盒的可能性最高
\end{document}