\documentclass{article}  
\usepackage{ctex}
\usepackage{amsmath}

\usepackage{geometry}

\geometry{a4paper,left=1cm,right=1cm,top=1cm,bottom=1cm}
\setlength{\parindent}{12pt} %自然段第一行的缩进量为12pt
\setlength{\parskip}{10pt plus1pt minus1pt}
%自然段之间的距离为10pt,并可在8pt到11pt之间变化
\setlength{\baselineskip}{8pt plus2pt minus1pt}
%行间距为8pt,并可在7pt到10pt之间变化
\setlength{\textheight}{21true cm}%版面高为21厘米
\setlength{\textwidth}{14.5true cm}%版面宽为14.5厘米



\begin{document}
    1.已知今日有雨明日也有雨的概率为0.7。今日无雨明日有雨的概率为0.5。\\
    可得转移矩阵
    \begin{gather*}
        \begin{matrix}
                    & \text{明日无雨} & \text{明日有雨}\\
            \text{今日无雨} & 0.5 & 0.5\\
            \text{今日有雨} &0.3 & 0.7    
        \end{matrix}
        \\
        \text{要求星期一有雨,星期三也有雨的概率,即求} P_{1,1}^{2}
        \text{,计算可得} P_{1,1}^{2} = 0.64
    \end{gather*}
    2.由题意可知转移矩阵为状态为$0 \sim n$ 的 $n \times n$ 矩阵。且有$p_{0,0} = 1,p_{n, n-1} = 1$。转移矩阵描述如下:\\
    \begin{gather*}
        \begin{matrix}
            1 & 0   & 0   & 0   & \cdots & 0  & 0 & 0 \\
            p & 0   & 1-p & 0   & \cdots & 0  & 0 & 0 \\
            0 & p   & 0   & 1-p & \cdots & 0  & 0 & 0  \\
            \vdots & \vdots & \vdots & \vdots & \vdots & \vdots & \vdots & \vdots\\
            0 & 0   & 0   & 0   &  \cdots & p & 0 & 1-p \\
            0 & 0   & 0   & 0   &  \cdots & 0 & 1 & 0
        \end{matrix}
    \end{gather*}
    令$A_0$表示最终落入状态0 的事件,记$q_{i,0} = \mathrm{Pr}(A_0|X_1 = i)$,当$i=0$时,$q_{0,0} = 1$\\
    \begin{gather*}
        q_{i,0} = \mathrm{Pr}(A_0|x_1 = i) 
        = \sum_{j=0}^{n} \mathrm{Pr}(A_0 | x_1 = i, x_2 = j) \cdot \mathrm{Pr}(x_2 =j|x_1=i)
        = \sum_{j=0}^{n} \mathrm{Pr}(A_0 | x_1 = j) \cdot p_{i,j}\\
    \end{gather*}
    该方程组表示为如下的矩阵\\
    \begin{gather*}
        \begin{matrix}
            1  & p-1 & 0   & 0   & \cdots & 0  & 0  & 0 & p \\
            -p & 1   & p-1 & 0   & \cdots & 0  & 0  & 0 & 0 \\
            0  & -p  & 1   & p-1 & \cdots & 0  & 0  & 0 & 0  \\
            \vdots & \vdots & \vdots & \vdots & \vdots & \vdots & \vdots & \vdots & \vdots\\
            0 & 0   & 0   & 0   &  \cdots & -p & 1  & p-1 & 0 \\
            0 & 0   & 0   & 0   &  \cdots & 0  & -1 & 1 & 0
        \end{matrix}
    \end{gather*}
    解得$q_{n,0} = q_{n-1, 0} = q_{n-2, 0} = \cdots = q_{1,0}$ \\
    代入$q_{1,0} + (p-1)q_{2, 0} = p$, 解得$q_{n,0} = q_{n-1, 0} = q_{n-2, 0} = \cdots = q_{1,0} = 1$\\
    $\mathrm{Pr}(A_0) = \sum_{i=1}^n \mathrm{Pr}(A_0|x_1=i) p(x_1=i)= \frac{1}{n}\sum_{i=1}^n \mathrm{Pr}(A_0|x_1=i) = 1$\\
    所以蚂蚁被吃掉的概率为1\\
    3.已知蜘蛛起始位置在0,蜘蛛的转移矩阵为
    $\begin{smallmatrix}
            0.7 & 0.3 \\
            0.3 & 0.7 
        \end{smallmatrix}$  
    \begin{gather*}
        p_{0,0}^{(n)} = \sum_i p_{0,i}^{(1)} p_{i,0}^{(n-1)} = \sum_i p_{0,i} (\sum_j p_{i,j}^{(1)} p_{j,0}^{(n-2)})
        = \sum_i \sum_j p_{0,i} p_{i,j} p_{j,0}^{(n-2)} = \cdots = \sum \cdots \sum p_{0,i} p_{i,j} \cdots p_{k,0}\\
        = \sum_{k = 0}^{\frac{n}{2}}C_{n}^{2k } (0.3)^{2k} \times (0.7)^{n - 2k}\\
        p_{0,1}^{(n)} = 1 - p_{0,0}^{(n)} = \sum_{k=1}^{\frac{n}{2}} C_{n}^{2k-1} (0.3)^{2k-1} \times (0.7)^{n-2k + 1}\\    
    \end{gather*}
    蚂蚁的起始位置在1,蚂蚁的转移矩阵为
    $\begin{smallmatrix}
        0.3 & 0.7 \\
        0.7 & 0.3 
    \end{smallmatrix}$
    \begin{gather*}
        p_{1,0} = \sum_{k=1}^{\frac{n}{2}} C_{n}^{2k-1} (0.7)^{2k-1} \times (0.3)^{n - 2k + 1}\\
        p_{1,1} = \sum_{k=0}^{\frac{n}{2}} C_{n}^{2k} (0.7)^{2k} \times (0.3)^{n-2k}
    \end{gather*}
\end{document}