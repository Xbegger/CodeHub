\documentclass{article}  
\usepackage{ctex}
\usepackage{amsmath}

\usepackage{geometry}

\geometry{a4paper,left=1cm,right=1cm,top=1cm,bottom=1cm}
\setlength{\parindent}{12pt} %自然段第一行的缩进量为12pt
\setlength{\parskip}{10pt plus1pt minus1pt}
%自然段之间的距离为10pt,并可在8pt到11pt之间变化
\setlength{\baselineskip}{8pt plus2pt minus1pt}
%行间距为8pt,并可在7pt到10pt之间变化
\setlength{\textheight}{21true cm}%版面高为21厘米
\setlength{\textwidth}{14.5true cm}%版面宽为14.5厘米



\begin{document}
    已知顾客到来过程为Possion分布,平均4人/h, $\lambda = 4$\\
    修理时间服从负指数分布,平均需要6min, $\mu = 10$\\
    (1)求修理店空闲的概率\\
    \begin{gather*}
        \rho = \frac{\lambda}{\mu} = 0.4\\
        p_0 = \left(\sum_{n=0}^\infty \rho^n\right)^{-1} 
    = \left(\frac{1}{\rho}\right)^{-1}
    = 1-\rho = 0.6\\
    \end{gather*}
    (2)店内恰有3个顾客的概率\\
    \begin{gather*}
        p_3 = \rho^3 p_0 = 0.4^3 \times 0.6 = 0.0384
    \end{gather*}
    (3)店内至少有一个顾客的概率\\
    \begin{gather*}
        P(\text{至少有一个顾客})= \sum_{n=1}^{\infty} p_n = 1- p_0 = 0.4
    \end{gather*}
    (4)店内平均顾客数\\
    \begin{gather*}
        L_s = \frac{\lambda}{\mu - \lambda}
    = \frac{4}{10 - 4} = \frac{2}{3} 
    \end{gather*}
    (5)每位顾客在店内平均逗留的时间\\
    \begin{gather*}
        W_s = \frac{1}{\mu - \lambda}
    = \frac{1}{10 - 4} = \frac{1}{6}
    \end{gather*}
    (6)等待服务的平均顾客数\\
    \begin{gather*}
        L_q = \frac{\lambda ^2}{\mu \times (\mu - \lambda)}
    = \frac{4^2}{10 \times (10 - 4)}
    = \frac{4}{15}
    \end{gather*}
    (7)每位顾客平均等待服务时间\\
    \begin{gather*}
        W_q = \frac{\lambda}{\mu(\mu - \lambda)}
    = \frac{4}{10 \times (10 - 4)} = \frac{1}{15}
    \end{gather*}
    (8)顾客在店内等待时间超过10min的概率\\
    \begin{gather*}
        \text{已知顾客在系统中的逗留时间T,服从$\mu-\lambda$的负指数分布}\\
        \text{取服务时间 $t \sim \mu $的负指数分布}\\
        P(T_q > 10) = \int_t P(T_q + t > 10 + t) \cdot P(t) 
    = \int_t P(T > 10 + t) \cdot P(t)
    = \int_t e^{-(\mu-\lambda)(10+t)} \times e^{-\mu t}
    = \frac{e^{-60}}{16}                            
    \end{gather*}
\end{document}